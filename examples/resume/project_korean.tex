%-------------------------------------------------------------------------------
%    SECTION TITLE
%-------------------------------------------------------------------------------
\cvsection{Project}


%-------------------------------------------------------------------------------
%    CONTENT
%-------------------------------------------------------------------------------
\begin{cventries}

%-eDL---------------------------------------------------------------
  \cventry
    {Software Engineer, Data Scientist} % Role
    {eDataLyzer 개발} % Event
    {서울시 서초구} % Location
    {2013.11 - 2019.8, 5년 10개월,\newline 30 M/M 이상} % Date(s)
    {
      \begin{cvitems} % Description(s)
        \item {eDataLyzer는 반도체의 웨이퍼 수율맵 분류와 유발요인을 상관관계 분석을 통해 찾는 제품입니다.}
        \item {이 프로젝트의 목표는 eDataLyzer를 빅데이터용으로 재개발하는 것입니다.}
        \item {그래서 저희는 이 프로젝트를 3가지 방법으로 진행했습니다.}
        \item {첫번째, 아키텍처를 모놀리틱에서 마이크로 서비스로 재구성했습니다.}
        \item {두번째, 단일팀을 역할 기반의 팀으로 재조직했습니다.(Client, Server, Algorithm, Research, Technical Sales/Support)}
        \item {세번째, C\#만 사용하는 것이 아니라 C\# + Java로 재개발했습니다.}
        \item {저는 Algorithm팀 소속으로, 주로 알고리즘을 빅데이터 신기술을 이용하여 병렬처리하는 일에 집중했습니다.}
        \item {간략하게, 3가지 방법으로 알고리즘을 병렬처리했습니다.}
        \item {첫번째, 스몰데이터용 알고리즘을 재개발했습니다.(Java, PostgreSQL, Spring)}
        \item {두번째, 하둡기반이 아닌 기술로 빅데이터용 알고리즘을 재개발했습니다.(GreenPlumDataBase PL/Java, Oracle-R)}
        \item {세번째, 하둡기반의 기술로 빅데이터용 알고리즘을 재개발했습니다.(Hadoop, BDA, Hawq, HBase, Spark, Eco system)}
        \item {이 프로젝트로 저희는 아래와 같이 많은 반도체 고객을 확보했습니다.}
        \item {한국(삼성전자, SK하이닉스, SK실트론), 일본(도시바, 샤프), 대만(TSMC), 중국(BOE)}
        \item {이 프로젝트의 유형은 PoC, pilot, production로 다양하게 확장되었습니다.}
      \end{cvitems}
    }
%-------------------------------------------------------------------

%-RL----------------------------------------------------------------
  \cventry
    {Researcher, Software Engineer} % Role
    {강화학습을 반도체 생산공정 관리에 적용하는 연구} % Event
    {서울시 서초구} % Location
    {2018.12 - 2019.8, 9개월, 6 M/M} % Date(s)
    {
      \begin{cvitems} % Description(s)
        \item {이 프로젝트의 유형은 제품화 전단계의 연구입니다.}
        \item {아래와 같은 이유로 8퍼즐을 강화학습 환경으로 선택했습니다.}
        \item {첫째, 팀원과 협업하기 위해서 Graph Theory를 쉽게 적용할 수 있는 일반화된 환경이 필요했다.}
        \item {그래서 반도체 제조 제어공정을 선택했다.}
        \item {둘째, 복잡하지 않고 적당한 환경을 찾기 위해서 반도체 제조 제어공정을 간소화한 8퍼즐을 선택했다.}
        \item {수율, 생산성, 안정성, 자동화율, 실시간성 등을 포기하고 최소경로만 집중했다.}
        \item {최근 연구현황은 다음과 같다.}
        \item {첫째, 8퍼즐을 Dynamic Programming으로 풀고, 다른 알고리즘(Shortest-path tree, Dijkstra)과 비교 분석했다.}
        \item {둘째, 8퍼즐을 QLearning, Deep SARSA, Polish Gradient로 풀지 못했다.}
        \item {연구의 남은 일은 왜 풀지 못했는지 찾고, 어떻게 극복할지 찾는 것이다.}
      \end{cvitems}
    }
%-------------------------------------------------------------------

%-MP----------------------------------------------------------------
  \cventry
    {Software Engineer} % Role
    {자동차 생산라인의 운송시스템 고장예측을 위한 Matrix Profile을 개발} % Event
    {서울시 서초구} % Location
    {2017.7 - 2017.12, 6개월, 2 M/M} % Date(s)
    {
      \begin{cvitems} % Description(s)
        \item {이 프로젝트의 목적은 시계열 데이터를 기반으로 모터의 정지를 예측하는 것입니다.}
        \item {고객의 생산직원이 일년에 한번씩 모터정지를 발견하고 예측을 요청했습니다.}
        \item {그러나 고객의 사무직원과 기존 프로젝트의 알고리즘도 정지를 예측하지 못했습니다.}
        \item {왜냐하면 그 알고리즘은 회전체 기계의 진동 분석에 집중했기 때문입니다.}
        \item {그래서 저희는 기존 프로젝트의 서브 프로젝트를 생성하기로 결정했고, 아래처럼 진행했습니다.}
        \item {첫째, 시계열 데이터를 위한 적합한 알고리즘으로 Matrix Profile을 선택했습니다.}
        \item {둘째, 파이썬으로 해당 알고리즘을 구현하고 고객사에 배표했고 정지예측문제를 풀었습니다.}
        \item {셋째, 자바로 구현하고 UI와 연동시키고, 고객이 저희 제품을 통해 정지예측문제를 해결하도록 교육했습니다.}
        \item {이 프로젝트의 고객은 현대기아자동차이고, 이 프로젝트의 유형은 PoC와 Pilot입니다.}
      \end{cvitems}
    }
%--------------------------------------------------------------------

%-PdM----------------------------------------------------------------
  \cventry
    {Researcher, Software Engineer} % Role
    {반도체 장비를 위한 예지정비를 개발} % Event
    {서울시 서초구} % Location
    {2016.11 - 2017.3, 5개월, 4 M/M} % Date(s)
    {
      \begin{cvitems} % Description(s)
        \item {이 프로젝트의 목적은 반도체 에칭 장비에 대한 예지정비를 개발하는 것입니다.}
        \item {고객의 경험적 정비(조건, 시간)를 바꾸기 위해서, 아래와 같이 이 프로젝트를 진행했습니다.}
        \item {첫째, 입력 데이터를 Self Organizing Map으로 재정의했고, health score를 입력 데이터의 벡터와 observation 벡터와의 거리로 정의했습니다.}
        \item {둘째, Double Exponential Weighted Moving Average를 health score에 적용하고, 입력 데이터의 벡터들에 대한 잔존생존시간을 구했습니다.}
        \item {그런데 고객에게 저희 제품이 너무 느리다는 feedback을 받았습니다. 그래서 추가적으로 이 프로젝트를 아래와 같이 진행했습니다.}
        \item {첫째, 제품의 병목현상을 DEWMA에서 발견했고, SOM에서는 발견하지 못했습니다.}
        \item {둘째, DEWMA에 Spark와 HDFS를 적용했습니다. 그리고 tuning point를 찾았습니다.}
        \item {이 프로젝트의 고객은 SKHynix이고, 프로젝트의 유형은 Pilot입니다.}
      \end{cvitems}
    }
%--------------------------------------------------------------------

%-QA-----------------------------------------------------------------
  \cventry
    {인턴쉽} % Role
    {새 제품의 품질보증 및 문서화 작업} % Event
    {성남시 분당구} % Location
    {2013.8 - 2013.10, 3개월, 5 M/M} % Date(s)
    {
      \begin{cvitems} % Description(s)
        \item {인턴쉽기간동안 아래와 같은 활동을 했습니다.}
        \item {첫째, 새 제품의 각 기능별로 기능적/비기능적 품질요소를 확인했습니다.}
        \item {둘째, 기존 문서를 사용자 관점에서 수정했습니다.s}
      \end{cvitems}
    }
%--------------------------------------------------------------------

\end{cventries}
